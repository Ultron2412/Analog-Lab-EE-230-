\documentclass[12pt]{article}
\usepackage{graphicx}
\usepackage{circuitikz}
\usepackage{float}

%
% Title.
\title{\textbf{
Experiment 1} \\ \textbf{Characterization of a CMOS
Inverter} }
% Author
\author{Amit Kumar - 16D070034}

% begin the document.
\begin{document}

% make a title page.
\maketitle

% section 1: overview.
\section{Overview of the experiment
}
The CMOS inverter is the simplest static complementary CMOS logic gate and also an essential component of all logic gates . In this experiments we characeterized CMOS and observed its DC \& AC behaviour, particularly we observed following -
\begin{itemize}
\item DC Transfer Characteristics
\item Output Characteristics
\item Delay Characteristics
\item Delay variation with supply voltage
\item Current drawn by the Ring Oscillator
\end{itemize}

For transfer \& output characteristics hex inverter IC {\em MM74C04} was used and for ring measurement tasks, a ring oscillator with 17 inveters was used to obtain a good estimate of the delay.

% section 2: setup/approach.
\section{Experiment setup and Obeservation}

Each part of the experiment essentially involves the CMOS inverter (IC MM74C04) as the elementary unit.The delay of a single CMOS inverter as their order of magnitude is very smallto measure and that is why a ring oscillator with 17 inverters was used . The Opamp buffer (TL072) was introduced so that the DSO does not add extra load to the circuit during the measurement.
\\ In further section, Circuit layouts for the various parts of the experiment are shown with relevant details.\\ \\
\textbf{Components used :} \\
IC MM74C04 $*$4, TL072, Decoupling Capacitors (0.1 $\mu$F) $*$2,\\20 K$\Omega$ potentiometer, Resistors(1.2K$\Omega$,1$\Omega$))


%You can easily include a diagram/picture as shown in Figure \ref{fig:demofig}.
%\begin{figure}
  % will center the figure.
 % \centering
  % include graphics (can include eps, jpg, pdf ...)
 % \includegraphics[scale=0.9]{figs/rectangle.eps}  % change scale factor to re-size the image.
  % give a caption.
  %\caption{A rectangle.}
  % a label to refer to the figure
  %\label{fig:demofig}
%\end{figure}

\subsection{DC Transfer Characteristics}
The transfer characteristic of the CMOS inverter is the plot of the output voltage as a function of the input voltage. As we vary the input voltage from 0V to V$_{DD}$, the output voltage will change from V$_{DD}$ to 0V.
The point in the transfer characteristics where the input and output voltages equal is called the \textit{Switching point}. This is an important factor for any switching circuit. The observation is tabulated in Table 1 and the characteristics have been plotted in Figure 1.\\From these observed values we get that \textbf{V$_{SW}$  $\approx$ 2.56V}.\\
But as we know that,


\begin{equation}
 V_{SW} = \frac{\sqrt{\beta_p}(V_{DD} -V_t) +\sqrt{\beta_n}V_t }{\sqrt{\beta_p}+\sqrt{\beta_n}}  
\end{equation}
\linebreak
Substituting the values we get \(\frac{\beta_p}{\beta_n}\) = 1.3

%code for figure
\begin{figure}
\includegraphics[width=12cm]{Trans_char.jpg}
\centering
\caption{The observed transfer characteristics of the CMOS inverter}
\end{figure}

\begin{table}[H]
\centering
\begin{tabular}{| c | c |} 
\hline
\textbf { Vin (V)} &\textbf{ Vout (V)} \\ 
\hline
0.5 & 4.98  \\
1 & 4.98  \\
1.5 & 4.97  \\
2 & 4.88  \\
2.1 & 4.83  \\
2.3 & 4.63  \\
2.5 & 2.67  \\
2.6 & 0.26  \\
2.7 & 0.17  \\
2.8 & 0.12  \\
3 & 0.03  \\
3.2 & 0.02  \\
3.5 & 0 \\
4 & 0 \\
5 & 0 \\
 \hline
\end{tabular}
\caption{DC Transfer Characteristics of the CMOS inverter}

\end{table}

\subsection{Output Characteristics}
In this part we try to measure the output characteristics of the CMOS inverter. The figures 2 and 3 represent the output characteristics of the device.
% circuit arrangement
\begin{figure}[H]
\includegraphics[width=12cm]{output_at_low.jpg}
\centering
\caption{Observed nature of the output characteristics at low input}
\end{figure}
 we can also conclude from the linear characteristics of the graphs that the output impedance of the device is constant. The observed values are given in Table 2 and 3.
\begin{figure}[H]
\includegraphics[width=12cm]{output_high.jpg}
\centering
\caption{Observed nature of the output characteristics at high input}
\end{figure}
\begin{table}[H]
\centering
\begin{tabular}{| c | c |} 
\hline
\textbf{ Vout (V)} &\textbf{ Iout (mA)} \\ 
\hline
3.25  & 2.75  \\
3.41  & 2.59  \\
3.52  & 2.49  \\
3.74  & 2.21  \\
3.87  & 2.05  \\
4.05  & 1.77  \\
4.22  & 1.51  \\
4.38  & 1.24  \\
4.46  & 1.08  \\
4.55  & 0.9 \\
\hline
\end{tabular}
\caption{Output characteristics at low input}

\end{table}

\begin{table}
\centering
\begin{tabular}{| c | c |} 
\hline
\textbf{ Vout (V)} &\textbf{ Iout (mA)} \\ 
\hline

0.03  & 0.229 \\
0.1 & 0.538 \\
0.21  & 1.012 \\
0.31  & 1.454 \\
0.41  & 1.839 \\
0.46  & 2.02  \\
0.51  & 2.28  \\
0.63  & 2.57  \\
0.68  & 2.75  \\
\hline
\end{tabular}
\caption{Output Characteristics at high input}
\label{table:demotable}
\end{table}

\subsection{Delay Characteristics of the CMOS Inverter}
In any digital circuit, an important characteristic of any device is the delay caused by it.
However, this delay is very small for each independent inverter and hence would require a very sophisticated equipment for measurement. In this part of the experiment we have achieved this task by using a 17- stage ring oscillator.  
\par The delay in this circuit can be modelled as-
\begin{equation}
    d_{abs}= k_o + k_{1}C_{load}
\end{equation}
Here, $k_o$ and $k_1$ are constants, $C_{load}$ is the load capacitance being driven by
the inverter, and $d_{abs}$ is the delay measured in seconds. It is convenient to express the $C_{load}$ as a multiple of the input capacitance $C_{in}$ of the inverter. The equation (2) can be re-written as-
\begin{equation}
    d_{abs}= k_o + \tau_{inv}\frac{C_{load}}{C_{in}}
\end{equation}
The delay can be expressed in multiples of $\tau_{inv}$ as
\begin{equation}
    d_{inv}= p_{inv} + \frac{C_{load}}{C_{in}}
\end{equation}
where, $p_{inv} = \frac{k_o}{\tau_{inv}}$\\
\par In table 4, the observed data for the variation of frequency or time period with load is presented. The results are summarized in the figure 4 and the snapshots are presented in Table 5.
\begin{table}[H]
\centering
\begin{tabular}{| c | c | c |} 
\hline
\textbf{Number of Load} & \textbf{Time Period ($\mu$s)}\\ 
\hline
1 & 1.087 \\
2 & 1.103 \\
3 & 1.12  \\
4 & 1.135 \\
5 & 1.15  \\
6 & 1.168 \\
7 & 1.186 \\
\hline
\end{tabular}
\caption{Variation of time period with number of loads}

\end{table}
\begin{figure}
\includegraphics[width=12cm]{load_vs_oscillation.jpg}
\centering
\caption{Variation of the time period of oscillation with respect to the number of loads at V$_D_D$= 5V}
\end{figure}

\begin{table}
\centering  % table will be centered.
\begin{tabular}{|c | c|} 
\includegraphics[width=7cm]{load2.jpg}


& 

\includegraphics[width=7cm]{load3.jpg}
\\
\includegraphics[width=7cm]{load4.jpg}


& 

\includegraphics[width=7cm]{load5.jpg}
\\
\includegraphics[width=7cm]{load6.jpg}

&

\includegraphics[width=7cm]{load7.jpg}
\end{tabular}
\caption{Snapshots at different number of loads VDD=5V}
\end{table}

The time period of oscillation is given by
\begin{equation}
    \tau_{inv}\times(34p_{inv}+(32+(2\times(1+Additionaloutputload)))
\end{equation}
As the rise and fall times are equal we have,\\
\begin{equation}
 Slope = 2\tau_{inv} = 0.019 \mu s
\end{equation}
\begin{equation}
 Intercept= 34\tau_{inv}(p_{inv} +1) = 1.07
\end{equation}
 \begin{equation}
 \tau_{inv}= 0.008 \mu s
 \end{equation}
 \begin{equation}
  p_{inv}= 2.13  
 \end{equation}
 \subsection{Delay variation with Supply Voltage}
The delay of the CMOS inverter varies with the supply voltage $V_{DD}$. This dependency is given below-

\begin{equation}
\textit{Time Period} \  \alpha \ \frac{V_{DD}}{(V_{DD}-V_t)^2}
\end{equation}
\\
 This part of the experiment is meant to verify this relation experimentally. The observed values are presented in Table 6.\\
\begin{table}[H]
\centering
\begin{tabular}{| c | c |} 
\hline
\textbf{ V$_{DD}$ (V)} &\textbf{ Time Period ($\mu$s)} \\ 
\hline
3 & 2.42  \\
3.5 & 2.112 \\
4 & 1.612 \\
4.5 & 1.325 \\
5 & 1.09  \\
5.5 & 0.956 \\
6 & 0.861 \\
\hline

\hline
\end{tabular}
\caption{Variation of time period of oscillation with Vdd at load=2}
\end{table}

\begin{figure}[H]
\includegraphics[width=12cm]{period_vs_supply.jpg}
\centering
\caption{Variation of time period of oscillation with sipply voltage(V$_D_D$) at load = 2}
\end{figure}
\begin{table}[H]
\centering  % table will be centered.
\begin{tabular}{|c | c|} % 3 columns, with text centered in each column, 
       %  the | specifies that there will be a line separating the
       %  adjacent columns.
%\hline  % horizontal line spanning the columns.
%Load & Frequency(MHz) & Time Period (ns) \\  % table entry 1, separated by &, ended by \\
%\hline  % horizontal line spanning the columns.

\includegraphics[width=8cm]{vdd4.jpg}


& 

\includegraphics[width=8cm]{vdd5.jpg}
\\
\includegraphics[width=8cm]{vdd6.jpg}


%& 

%\includegraphics[width=8cm]{save.jpeg}

\\

%\hline % horizontal line.
\end{tabular}
\caption{Snaps of DSO for load = 2 at V$_D_D$ 4V, 5V and 6V respectively }
\label{table:demotable}
\end{table}

\subsection{Current drawn by the Ring Oscillator}
As in the CMOS technology the steady power loss is zero and so is the current and so when the inverter output switches from low to high or vice versa,it draws current from the power supply.In the experimental set up, the current drawn by the ring oscillator is the sum of the currents drawn by the individual inverters which can be seen in the snapshot of the DSO below.
\begin{figure}[H]
\includegraphics[width=12cm]{final.jpg}
\centering
\caption{Snapshot of the DSO measuring the switching current and Vout}
\end{figure}
\\
\par The measured values of \textless \textbf{I} \textgreater = 2.1 mA and $\textbf{I_{P-P}}$=20.9 mA.


\end{document}